\documentclass[11pt,a4paper,uplatex,dvipdfmx]{ujarticle} 		% for uplatex
%\documentclass[11pt,a4j,dvipdfmx]{jarticle} 					% for platex
\input{pieces/form00_header} % pieces
\input{pieces/kakenhi7} % pieces
\input{pieces/form01_header} % pieces
\input{pieces/form02_2021_header} % pieces
\input{pieces/hook3} % pieces
%#Name: gakuhen_a_koubo
\input{pieces/form04_jsps_headers} % pieces
%form04_mext_headers_2020
%%\setlength{\oddsidemargin}{-18pt}
%%\setlength{\evensidemargin}{-18pt}
%%\setlength{\textwidth}{486pt}
%%
\newcommand{\mextPageUnderline}{\underline}	
% shingaku_field has no underline under the page #, but others do.  Sigh...
% This will be redefined in form04_shingaku_field_header.tex.	2017-09-03 Taku

\newcommand{\mextPageLF}{}
% page # for shingaku_keikaku is at the bottom of the header.
% page # for shingaku_koubo is one line above the header bottom.

% How come MEXT forms have no unified format??  It is a matter of sense of beauty...
\fancypagestyle{mext-p1-sameline-header}{%
	\fancyhf{}
	\fancyhead[L]{\headerfont\textbf{\様式 (応募内容ファイル (添付ファイル項目))}}
	\fancyhead[R]{\headerfont\textbf{\mextPageUnderline{\研究種目header−\KLJInt{\thepage}}}}
}

\fancypagestyle{mext-p1-offset-header}{%
	\fancyhf{}
	\fancyhead[L]{\headerfont\textbf{\様式 (応募内容ファイル (添付ファイル項目))}\\
			\rule{0pt}{0pt}\mextPageLF}
	\fancyhead[R]{\headerfont\textbf{\mextPageUnderline{\研究種目header−\KLJInt{\thepage}}}\mextPageLF}
}

\fancypagestyle{mext-p1-offset2-header}{%
	\fancyhf{}
	\fancyhead[L]{\headerfont\textbf{\様式 (応募内容ファイル (添付ファイル項目))}\\
			\rule{0pt}{15pt}\\\mextPageLF}
	\fancyhead[R]{\headerfont\textbf{\mextPageUnderline{\研究種目header−\KLJInt{\thepage}}}\mextPageLF}
}

\newcommand{\MEXTVeryFirstPageStyle}[5]{%
%	Defines the header for the very first page of the form.
%	Called from \KLShumokuFirstPageStyle in form04_***.
%--------------------------------
%	#1: page style name
%	#2: 様式 研究計画調書 (**)
%	#3: 研究種目名
%	#4: 項目名
%	#5: sectionNo
%--------------------------------
	\fancypagestyle{JSPSVeryFirstPageStyle}{% The name is not taken from #1, because 
		\fancyhf{}
		\fancyhead[L]{\hspace{-37pt}\headerfont#2\ (添付ファイル項目)\\
				\rule{0pt}{18pt}\\}
%				\rule{0pt}{0pt}\\}
		\fancyhead[R]{\headerfont\textbf{#3\ \KLJInt{\thepage}}\vspace{-5pt}\\
			\rule{0pt}{0pt}\\}
%		\fancyhead[R]{\headerfont\textbf{#3\ \KLJInt{\thepage}\\}}
	}
	\thispagestyle{JSPSVeryFirstPageStyle}
}



\newcommand{\MEXTFieldVeryFirstHeaderWithSectionNo}[5]{%
%	Defines the the very first header for MEXT shingaku_field form in form XXXX-N-n.
%--------------------------------
%	#1: page style name
%	#2: 様式
%	#3: 研究種目名
%	#4: 項目名
%	#5: sectionNo
%--------------------------------
	\fancypagestyle{MEXTFieldVeryFirstHeaderWithSectionNo}{%
		\fancyhf{}
		\fancyhead[L]{\headerfont\textbf{#2(「領域計画書」応募内容ファイル(添付ファイル項目))}\\
				\rule{0pt}{18pt}\\}
		\fancyhead[R]{\headerfont\textbf{\mextPageUnderline{#3−\KLJInt{#5}−\KLJInt{\thepage}}}\mextPageLF}
	}
	\thispagestyle{MEXTFieldVeryFirstHeaderWithSectionNo}
}

\newcommand{\MEXTDefaultHeaderWithSectionNo}[5]{%
%	Defines the default header for MEXT forms in form XXXX-N-n.
%--------------------------------
%	#1: page style name
%	#2: 様式
%	#3: 研究種目名
%	#4: 項目名
%	#5: sectionNo
%--------------------------------
	\fancypagestyle{MEXTDefaultHeaderWithSectionNo}{%
		\fancyhf{}
		\fancyhead[R]{\headerfont\textbf{\mextPageUnderline{#3−\KLJInt{#5}−\KLJInt{\thepage}}}\mextPageLF}
	}
	\pagestyle{MEXTDefaultHeaderWithSectionNo}
}

\newcommand{\MEXTDefaultHeaderWithParenthesis}[5]{%
%	Defines the default header for MEXT forms in form XXXX-N-(n).
%--------------------------------
%	#1: page style name
%	#2: 様式
%	#3: 研究種目名
%	#4: 項目名
%	#5: sectionNo
%--------------------------------
	\fancypagestyle{MEXTDefaultHeaderWithParenthesis}{%
		\fancyhf{}
		\fancyhead[R]{\headerfont\textbf{\mextPageUnderline{#3\KLJInt{#5}−(\KLJInt{\thepage})}}\mextPageLF}
	}
	\pagestyle{MEXTDefaultHeaderWithParenthesis}
}

\newcommand{\MEXTDefaultHeader}[5]{%
%	Defines the default header for MEXT forms in form XXXX-n.
%--------------------------------
%	#1: page style name
%	#2: 様式
%	#3: 研究種目名
%	#4: 項目名
%	#5: sectionNo
%--------------------------------
	\fancypagestyle{MEXTDefaultHeader}{%
		\fancyhf{}
		\fancyhead[R]{\headerfont\textbf{\mextPageUnderline{#3−\KLJInt{\thepage}}}\mextPageLF}
	}
	\pagestyle{MEXTDefaultHeader}
}

%%\newcommand{\MEXTDefaultPageStyle}[5]{%
%%%	Defines the default header for the subject.
%%%	Called from \KLShumokuDefaultPageStyle in form04_***.
%%%--------------------------------
%%%	#1: page style name
%%%	#2: 様式
%%%	#3: 研究種目名
%%%	#4: 項目名
%%%	#5: sectionNo
%%%--------------------------------
%%	\fancypagestyle{JSPSDefaultPageStyle}{%
%%		\fancyhf{}
%%		\fancyhead[L]{\headerfont\textbf{  【#4(つづき)\ 】}\vspace{-7pt}\\}
%%		\fancyhead[R]{\headerfont\textbf{#3\ \KLJInt{\thepage}}\vspace{-5pt}\\
%%			\rule{0pt}{0pt}\\}
%%%		\fancyhead[R]{\headerfont\textbf{#3\ \KLJInt{\thepage}\\}}	
%%        }
%%        \pagestyle{JSPSDefaultPageStyle}
%%}
%%
 % pieces
% form04_gakuhen_a_koubo_header.tex

\renewcommand{\mextPageLF}{}

\setlength{\headheight}{50pt}
\setlength{\topmargin}{-67pt}
\setlength{\textheight}{258mm}

% ===== Global definitions for the Kakenhi form ======================
% 基本情報
\newcommand{\様式}{様式S−74 研究計画調書(公募研究)}
\newcommand{\研究種目}{学術変革}
\newcommand{\研究種別}{(A)}
\newcommand{\研究種目後半}{公募研究}
%\ifthenelse{\isundefined{\研究種別}}{
%	\newcommand{\研究種別}{}
%}{}%
\newcommand{\研究種目header}{学術変革(A)\,(公募)}

\newcommand{\KLMainFile}{gakuhen\_a\_koubo.tex}
\newcommand{\KLYoshiki}{gakuhen_a_koubo_header}

%==========================================================
 % pieces
% ===== Global definitions for the Kakenhi form ======================
% 基本情報
%
%------ 研究課題名  -------------------------------------------
\newcommand{\研究課題名}{曲率ゆらぎの統計と原始ブラックホール量の精密対応}

%----- 研究機関名と研究代表者の氏名-----------------------
\newcommand{\研究機関名}{名古屋大学}
\newcommand{\研究代表者氏名}{多田祐一郎}
\newcommand{\me}{\underline{\underline{Y.~Tada}}} 
%---- 研究期間の最終年度 ----------------
\newcommand{\研究期間の最終元号年度}{4}  %令和で、半角数字のみ
%========================================

% inst_general.tex
%--------------------------------------------------------------------
% For writing instructions
%--------------------------------------------------------------------
\newcommand{\KLInstWOGeneral}[1]{%
	\noindent
 ーー ※留意事項 ーーーーーーーーーーーーーーーーーーーーーーーーーーーーーー\\
		#1\\
 ーーーーーーーーーーーーーーーーーーーーーーーーーーーーーーーーーーーーーーーー
}

\newcommand{\KLInst}[1]{%
	\noindent
	\ifthenelse{\equal{#1}{}}{%
 ーー ※留意事項 ーーーーーーーーーーーーーーーーーーーーーーーーーーーーーーー\\
	}{%
 ーー ※留意事項\textcircled{1} ーーーーーーーーーーーーーーーーーーーーーーーーーーーーーーー\\
		#1\\
		
	\noindent
 ーー ※留意事項\textcircled{2} ーーーーーーーーーーーーーーーーーーーーーーーーーーーーーーー\\
	}
}

% local variables for \GeneralInstructions ------------------------
\newcommand{\留意事項内種目名}{}			% #1
\newcommand{\留意事項内記入要領名}{}		% #2
\newcommand{\留意事項内種目名削除用}{}	% #3

\newcommand{\GeneralInstructions}[3]{%
	\ifthenelse{\equal{#1}{}}{%
		\renewcommand{\留意事項内種目名}{研究計画調書}%
	}{%
		\renewcommand{\留意事項内種目名}{#1}%
	}%
	\ifthenelse{\equal{#2}{}}{%
		\renewcommand{\留意事項内記入要領名}{作成・記入要領}%
	}{%
		\renewcommand{\留意事項内記入要領名}{#2}%
	}%
	\ifthenelse{\equal{#3}{}}{%
		\renewcommand{\留意事項内種目名削除用}{\留意事項内種目名}%
	}{%
		\renewcommand{\留意事項内種目名削除用}{#3}%
	}%
  1.作成に当たっては、\留意事項内種目名\留意事項内記入要領名 を必ず確認すること。\\
  2.本文全体は11ポイント以上の大きさの文字等を使用すること。\\
  3.各頁の上部のタイトルと指示書きは動かさないこと。\\
  4.指示書きで定められた頁数は超えないこと。なお、空白の頁が生じても削除しないこと。\\
  \textcolor{red}{5.本留意事項は、\留意事項内種目名削除用 の作成時には削除すること。(\texttt{\textbackslash JSPSInstructions}などを消す)}\\
 ーーーーーーーーーーーーーーーーーーーーーーーーーーーーーーーーーーーーーーーー
}


\newcommand{\PapersInstructions}{%
 ーー ※留意事項 ーーーーーーーーーーーーーーーーーーーーーーーーーーーーーーー\\
1. 研究業績(論文、著書、産業財産権、招待講演等)は、網羅的に記載するのではなく、\\
 本研究計画の実行可能性を説明する上で、その根拠となる文献等の主要なものを適宜記\\
 載すること。\\
2. 研究業績の記述に当たっては、当該研究業績を同定するに十分な情報を記載すること。\\
 例として、学術論文の場合は論文名、著者名、掲載誌名、巻号や頁等、発表年(西暦)、\\
 著書の場合はその書誌情報、など。\\
3. 論文は、既に掲載されているもの又は掲載が確定しているものに限って記載すること。\\
\textcolor{red}{4. 本留意事項は、\留意事項内種目名削除用 の作成時には削除すること。
 (\texttt{\textbackslash PapersInstructions}を消す)}\\
 ーーーーーーーーーーーーーーーーーーーーーーーーーーーーーーーーーーーーーーーー\\
} % pieces
%inst_gakuhen_a_koubo.tex
\newcommand{\JSPSInstructions}{%
	\\
	\KLInst{}
	\GeneralInstructions{研究計画調書(公募研究)}{}{}
}
 % pieces
% user07_header
% ===== my favorite packages ====================================
% ここに、自分の使いたいパッケージを宣言して下さい。
\usepackage{wrapfig}
\usepackage{amssymb}
%\usepackage{mb}
%\DeclareGraphicsRule{.tif}{png}{.png}{`convert #1 `dirname #1`/`basename #1 .tif`.png}
\usepackage{lineno}

\usepackage[dvipdfmx]{graphicx,xcolor}
\usepackage[multi,deluxe,bold,expert]{otf}
\usepackage[framemethod=tikz]{mdframed}
\usepackage[dvipdfmx
, colorlinks = true
, urlcolor = darkblue
, citecolor = darkblue
, linkcolor = darkblue]{hyperref}
\usepackage{cite}
\usepackage{url}
\usepackage{physics}
\usepackage{xspace}


% ===== my personal definitions ==================================
% ここに、自分のよく使う記号などを定義して下さい。
\newcommand{\klpionn}{K_L \to \pi^0 \nu \overline{\nu}}
\newcommand{\kppipnn}{K^+ \to \pi^+ \nu \overline{\nu}}

% ----- 業績リスト用 -------------
\newcommand{\paper}[6]{%
	% paper{title}{authors}{journal}{vol}{pages}{year}
	\item ``#1'', #2, #3 {\bf #4}, #5 (#6).			% お好みに合わせて変えてください。
}

\newcommand{\etal}{\textit{et al.\ }}
\newcommand{\ca}[1]{*#1}	% corresponding author;   \ca{\yukawa}  みたいにして使う
\newcommand{\invitedtalk}{招待講演}

\newcommand{\yukawa}{H.~Yukawa}					% no underline
%\newcommand{\yukawa}{\underline{\underline{H.~Yukawa}}}	% with 2 underlines
\newcommand{\tomonaga}{S.~Tomonaga}

\newcommand{\prl}{Phys.\ Rev.\ Lett.\ }		% よく使う雑誌も定義すると楽

\definecolor{monza}{HTML}{CF000F}
\definecolor{darkblue}{HTML}{00008b}
\definecolor{darkmagenta}{HTML}{8b008b}

%\definecolor{stochastic}{HTML}{FF7E79}
%\definecolor{stochastic}{HTML}{FCB9B9}
\definecolor{stochastic}{HTML}{FCD0D0}
%\definecolor{PBH}{HTML}{76D6FF}
\definecolor{PBH}{HTML}{D4F4FF}
%\definecolor{GW}{HTML}{73FA79}
\definecolor{GW}{HTML}{D4FFD4}

\renewcommand{\emph}[1]{{\sffamily\gtfamily\bfseries #1}}
\newcommand{\subject}[1]{\noindent{\sffamily\gtfamily\bfseries #1}~~}
\newcommand{\subsubject}[1]{\noindent \underline{#1}~~}
\newcommand{\stochastic}[1]{\noindent \colorbox{stochastic}{\ul{i) #1}}~~}
\newcommand{\PBH}[1]{\noindent\colorbox{PBH}{\ul{ii) #1}}~~}
\newcommand{\GW}[1]{\noindent\colorbox{GW}{\ul{iii) #1}}~~}
%\renewcommand{\bf}{\bfseries\sffamily\gtfamily}

\newcommand{\Red}[1]{\textcolor{monza}{\sffamily\gtfamily\bfseries #1}}
\newcommand{\Blue}[1]{\textcolor{darkblue}{\sffamily\gtfamily\bfseries #1}}

\newenvironment{footnoteSBL}{
	\baselineskip=10pt
}

\makeatletter
\renewenvironment{thebibliography}[1]
{\section*{\refname\@mkboth{\refname}{\refname}}%
\list{\@biblabel{\@arabic\c@enumiv}}%
{\settowidth\labelwidth{\@biblabel{#1}}%
\leftmargin\labelwidth
\advance\leftmargin\labelsep
\setlength\itemsep{0.2zh}%←ここの数値を調整(行間のつまり具合)
\setlength\baselineskip{11pt}%←ここの数値を調整(追加)(文字の大きさ)
\@openbib@code
\usecounter{enumiv}%
\let\p@enumiv\@empty
\renewcommand\theenumiv{\@arabic\c@enumiv}}%
\sloppy
\clubpenalty4000
\@clubpenalty\clubpenalty
\widowpenalty4000%
\sfcode`\.\@m}
{\def\@noitemerr
{\@latex@warning{Empty `thebibliography' environment}}%
\endlist}
\makeatother


% ===== 欄外メモ ==================
\newcommand{\memo}[1]{\marginpar{#1}}
%\renewcommand{\memo}[1]{}	% 全てのメモを表示させないようにするには、行頭の"%"を消す


%\input{../../sample/simple/contents}	% skip
\input{pieces/hook5} % pieces

\begin{document}

\mgfamily\sffamily

\input{pieces/hook7} % pieces
%#Split: 01_purpose_plan  
%#PieceName: p01_purpose_plan
% p01_purpose_plan_00.tex
\KLBeginSubjectWithHeaderCommands{01}{1}{1 研究目的、研究方法など}{4}{F}{}{\MEXTVeryFirstPageStyle}{\JSPSDefaultPageStyle}

\section{1 研究目的、研究方法など}
%    <<最大 4ページ>>

%s02_purpose_plan_with_abstract
\noindent
\textbf{(概要)}\\
%begin 研究目的及び研究計画の概要空行付き ====================
	
	\vspace*{10zw}	% (概要)と(本文)の間が10行程度になるよう、必要に応じて値を調整してください。	
%end 研究目的及び研究計画の概要空行付き ====================

\noindent
\rule{\linewidth}{1pt}\\
\noindent
\textbf{(本文)}
%begin 研究目的と研究計画	====================

%end 研究目的と研究計画	====================

\input{pieces/p01_purpose_plan_01}

%#Split: 02_background  
%#PieceName: p02_background
\input{pieces/p02_background_00}
\section{2 本研究の着想に至った経緯など}
%    <<最大 1ページ>>

%s03_background
%begin 本研究の着想に至った経緯など ====================

%end 本研究の着想に至った経緯など ====================

\input{pieces/p02_background_01}

%#Split: 03_abilities  
%#PieceName: p03_abilities
\input{pieces/p03_abilities_00}
\section{3 応募者の研究遂行能力及び研究環境}
%    <<最大 2ページ>>

% s14_abilities
%begin 応募者の研究遂行能力及び研究環境 ====================

\begin{mdframed}[roundcorner=0.5zw,
	%skipabove=1zw,skipbelow=1zw,
	innertopmargin=0.8zw,innerbottommargin=0.8zw,
	%innerleftmargin=0.8zw,innerrightmargin=0.8zw,
	%rightmargin=5000pt,leftmargin=50pt,
	linecolor=black!50,linewidth=0.2zw,
	backgroundcolor=black!10]
	{\bfseries\gtfamily\sffamily\large 1. これまでの研究活動}
\end{mdframed}

\noindent
申請者のこれまでの経歴や業績を以下に簡単にまとめた. 
申請者の web ページ \url{https://nekomammat.github.io/indexJP.html} も合わせて参照されたい.

\begin{description}\itemsep-1mm \itemindent4zw \labelwidth6zw
	\item[\gtfamily 職歴・フェローシップ]
	\item[\rm\sffamily 2019.04--] \emph{非常勤講師 (力学1, 2)} 大同大学
	\item[\rm\sffamily 2018.04--] \emph{日本学術振興会特別研究員 PD} 名古屋大学大学院 理学研究科 宇宙論研究室
	\item[\rm\sffamily 2017.04--] \emph{ポスドク研究員} \itemsep-2mm
	\item[\rm\sffamily\hfill 2018.03  ] Institut d'Astrophysique de Paris, France \itemsep-1mm
	\item[\rm\sffamily 2015.04--] \emph{日本学術振興会特別研究員 DC2} \itemsep-2mm
	\item[\rm\sffamily\hfill 2017.03  ] 東京大学 カブリ数物連携宇宙研究機構および宇宙線研究所 \itemsep-1mm
	\item[\rm\sffamily 2012.10--] \emph{フォトンサイエンス・リーディング大学院} \itemsep-2mm
	\item[\rm\sffamily\hfill 2017.03  ] 東京大学 カブリ数物連携宇宙研究機構および宇宙線研究所
\end{description}
\begin{description}\itemsep-1mm \itemindent4zw \labelwidth6zw
	\item[\gtfamily 採択・受賞歴]
	\item[\rm\sffamily 2019.02] \emph{若手代表発表者} FAPESP-JSPS Workshop on dark energy, dark matter, and galaxies
	\item[\rm\sffamily 2017.02.24] \emph{所長賞 (博士部門)} 第6回修士博士研究発表会, 宇宙線研究所
\end{description}
\begin{description}\itemsep-1mm \itemindent4zw \labelwidth6zw
	\item[\gtfamily 外部資金獲得状況]
	\item[\rm\sffamily 2019--2020] \emph{科学研究費助成事業 若手研究} \\
		JP19K14707「ストカスティック形式で迫る重力と量子論」1,560,000 円, 研究代表者
\end{description}
\begin{description}\itemsep-1mm \itemindent4zw \labelwidth6zw
	\item[\gtfamily 研究者活動]
	\item[\hfill -] \emph{サイエンスメンバー} \itemsep-2mm
	\item[] International Research Network Extragalactic astrophysics and Cosmology (NECO) \itemsep-1mm
	\item[\hfill -] \emph{査読} EPJC, PTEP, JCAP, PRD, Universe
	\item[\rm\sffamily 2014.10.01--] \emph{留学} ヘルシンキ大学 Kari Enqvist 教授 \itemsep-2mm
	\item[\rm\sffamily\hfill 12.22  ] フォトンサイエンス・リーディング大学院のコースワーク \itemsep-1mm
\end{description}

申請者はこれまで\ul{22本の論文}を発表し, \ul{26件の国際会議}, \ul{24件の国内外研究所でのセミナー発表}を行なっている.
2019年にはブラジルとの2国間研究会 ``FAPESP-JSPS Workshop on dark energy, dark matter, and galaxies"
にて\ul{若手代表発表者として選出}され, 原始ブラックホールに関する発表を行い,
2020年には ``Online JGRG Workshop" にてインフレーションの確率形式に関する発表について, \ul{Outstanding Presentation Award Gold Prize} を受賞した. 
フェローシップとして\ul{フォトンサイエンス・リーディング大学院}や\ul{日本学術振興会特別研究員 DC2 および PD} に採用され,
\ul{科学研究費助成事業若手研究 (1回目)} にも採択されている.
これら研究資金を利用して海外経験も積み, 学生の時から海外研究者とも共同研究を行ってきた. 
2017 年度は\ul{フランス国立科学研究センター (CNRS) に PD 研究員として採用}されパリ天体物理学研究所 (IAP) に滞在していたが, 
特別研究員 PD に採用されたため任期途中で切り上げ帰国することとなり, 現在は名古屋大学に所属している.
今年度採用分の\ul{名古屋大学高等研究院 YLC 特任助教として内定}され, 来年度からも引き続き名古屋大学宇宙論研究室に所属し,
安定した研究活動と大学院生の研究指導を行うことができる.
その他の詳細な経歴については申請者の web ページ \url{https://nekomammat.github.io/indexJP.html}も参照されたい.

研究テーマは主にインフレーションに関連した初期宇宙の物理についてである.
最初の論文~\cite{Fujita:2013cna} ではインフレーションの確率形式において曲率ゆらぎを直接計算する手法を提唱し,
続く論文~\cite{Fujita:2014tja} では本手法を実際に, ゆらぎが大きくなって従来手法が破綻する理論に適用し, 初めて曲率ゆらぎの定量的な計算に成功した.
論文~\cite{Kawasaki:2015ppx} では当理論における原始ブラックホール形成も議論し, 現実的な確率形式の利用法を確立してきたと言える.
フランスポスドク時代には, これまで多くの研究者が利用してきた素朴なインフレーション確率形式の定式化では, 数学的に不定性があり,
より一般的なインフレーション理論に対して運動方程式が1つに定まらない問題があることを発見した~\cite{Pinol:2018euk}.
日本に戻ってきてからもフランスの共同研究者らと研究を続け, 先日ついに数学的に一意で, 一般的なインフレーション理論に対しても適用可能な
確率形式の定式化に成功し~\cite{Pinol:2020cdp}, インフレーション理論の大規模包括的解析の準備が整ったと言える.

原始ブラックホールについては一連の研究~\cite{Kawasaki:2016pql,Inomata:2016rbd,Inomata:2017okj,Inomata:2017uaw} が業界から注目されている.
原始ブラックホールの実現には大きな曲率ゆらぎが必要になる一方, 大規模スケール ($\gtrsim1\,\mathrm{Mpc}$) においては曲率ゆらぎが非常に小さかったことが CMB 等の宇宙観測からわかっており,
したがってこれまで原始ブラックホールを実現するインフレーション理論にはパラメータの細かい調整を必要とする不自然なものが多かった.
一方我々はインフレーションが2回以上起き, 大規模スケールと原始ブラックホールスケールのゆらぎが異なるインフレーションに対応していたとすれば, 非常に簡単に原始ブラックホールを実現できることを指摘した.
このような多段階インフレーションは, 量子重力理論の候補である超弦理論の文脈から支持されることも議論している~\cite{Kogai:2020jkq}.
同じく名古屋大学宇宙論研究室所属の\emph{横山修一郎氏 (名古屋大学助教)} とはこのような多段階インフレーションにて, LIGO/Virgo の合体ブラックホール, 暗黒物質, そして OGLE グループが発見した銀河中心に向けての正体不明の重力レンズ現象~\cite{Niikura:2019kqi}
の全てを原始ブラックホールで同時に説明することができること示した~\cite{Tada:2019amh}.
横山氏とはゆらぎの統計性が原始ブラックホールの密集形成に影響することも指摘し~\cite{Tada:2015noa}, これは合体重力波の頻度の観点からも重要である.

これまで申請者は現研究室で大学院生の研究指導にも積極的に貢献してきたが~\cite{Kogai:2020jkq,Mikura:2020qhc,Abe:2020sqb},
論文~\cite{Abe:2020sqb} では大学院生らとともに, 誘導重力波が QCD 相転移プラズマの音速の変化に影響されることを示し, 
したがって誘導重力波の詳細な観測によって初期宇宙の QCD 相転移プラズマの性質を探り得る可能性があることを示唆した.
引き続き具体的な重力波観測計画を考慮し, プラズマパラメータの観測可能性を定量的に議論する研究を行う予定である.

以上が申請者のこれまでの研究活動と実績の簡単なまとめであり, 本研究課題の遂行可能性を示す客観的証拠であると言える.


\begin{mdframed}[roundcorner=0.5zw,
	%skipabove=1zw,skipbelow=1zw,
	innertopmargin=0.8zw,innerbottommargin=0.8zw,
	%innerleftmargin=0.8zw,innerrightmargin=0.8zw,
	%rightmargin=5000pt,leftmargin=50pt,
	linecolor=black!50,linewidth=0.2zw,
	backgroundcolor=black!10]
	{\bfseries\gtfamily\sffamily\large 2. 研究環境}
\end{mdframed}

\noindent
名古屋大学高等研究院の YLC 特任助教に内定されたため, 引き続き名古屋大学宇宙論研究室で安定した研究活動が可能である.
当研究室においてすでに研究者および大学院生らとの共同研究実績を挙げており, 本研究課題に対して研究環境が適していることを示している.
特に原始ブラックホールに関しては当分野を先導している横山氏や\emph{柳哲文氏 (名古屋大学講師)} の助けを借りることができるため, 非常に優れた研究環境だと言える.
確率形式の大規模計算に対して, 当研究室はクラスター計算機 ``galaxy" を所有しており研究に役立つであろう.
また本研究課題は国内においては\emph{藤田智弘氏 (東京大学研究員)} や\emph{徳田順生氏 (神戸大学研究員)} との共同研究を計画しており,
名古屋大学は各研究所に対し地理的に中心に位置しているため, 共同研究を進めやすい.
国外研究者としてはフランスの共同研究者である Vennin 氏や \emph{S\'ebastien Renaux-Petel 氏 (IAP 研究員)}, \emph{Lucas Pinol 氏 (IAP 大学院生)} との引き続きの共同研究はもちろん,
イギリスの Wands 氏や \emph{Christian Byrnes 氏 (Sussex 上級主任研究員)}, イタリアの \emph{Sabino Matarrese 氏 (Padova 教授)} や \emph{Nicola Bartolo 氏 (Padova 准教授)}
との議論および協力を予定している (実際直近では 2020 年 11 月に Padova グループでのオンラインセミナー発表を依頼されている).
普段は電子メールやオンライン会議, チャットツールを駆使して連絡をとっているが, より円滑な共同研究活動のためには旅費や設備準備費として本科学研究費が必要不可欠である.



\bigskip

\begin{thebibliography}{99}

\scriptsize % フォントサイズを下げる
\setlength{\itemsep}{-2pt} % 行間を縮める

%\cite{Fujita:2013cna}
\bibitem{Fujita:2013cna}
T.~Fujita, M.~Kawasaki, \me and T.~Takesako,
%``A new algorithm for calculating the curvature perturbations in stochastic inflation,''
JCAP \textbf{12}, 036 (2013).
%doi:10.1088/1475-7516/2013/12/036
%[arXiv:1308.4754 [astro-ph.CO]].
%33 citations counted in INSPIRE as of 29 Oct 2020

%\cite{Fujita:2014tja}
\bibitem{Fujita:2014tja}
T.~Fujita, M.~Kawasaki and \me,
%``Non-perturbative approach for curvature perturbations in stochastic $\delta N$ formalism,''
JCAP \textbf{10}, 030 (2014).
%doi:10.1088/1475-7516/2014/10/030
%[arXiv:1405.2187 [astro-ph.CO]].
%33 citations counted in INSPIRE as of 29 Oct 2020

%\cite{Kawasaki:2015ppx}
\bibitem{Kawasaki:2015ppx}
M.~Kawasaki and \me,
%``Can massive primordial black holes be produced in mild waterfall hybrid inflation?,''
JCAP \textbf{08}, 041 (2016).
%doi:10.1088/1475-7516/2016/08/041
%[arXiv:1512.03515 [astro-ph.CO]].
%32 citations counted in INSPIRE as of 29 Oct 2020

%\cite{Pinol:2018euk}
\bibitem{Pinol:2018euk}
L.~Pinol, S.~Renaux-Petel and \me,
%``Inflationary stochastic anomalies,''
Class. Quant. Grav. \textbf{36}, no.7, 07LT01 (2019).
%doi:10.1088/1361-6382/ab097f
%[arXiv:1806.10126 [gr-qc]].
%17 citations counted in INSPIRE as of 29 Oct 2020

%\cite{Pinol:2020cdp}
\bibitem{Pinol:2020cdp}
L.~Pinol, S.~Renaux-Petel and \me,
%``A manifestly covariant theory of multifield stochastic inflation in phase space,''
[arXiv:2008.07497 [astro-ph.CO]].
%5 citations counted in INSPIRE as of 29 Oct 2020


%\cite{Kawasaki:2016pql}
\bibitem{Kawasaki:2016pql}
M.~Kawasaki, A.~Kusenko, \me and T.~T.~Yanagida,
%``Primordial black holes as dark matter in supergravity inflation models,''
Phys. Rev. D \textbf{94}, no.8, 083523 (2016).
%doi:10.1103/PhysRevD.94.083523
%[arXiv:1606.07631 [astro-ph.CO]].
%81 citations counted in INSPIRE as of 29 Oct 2020

%\cite{Inomata:2016rbd}
\bibitem{Inomata:2016rbd}
K.~Inomata, M.~Kawasaki, K.~Mukaida, \me and T.~T.~Yanagida,
%``Inflationary primordial black holes for the LIGO gravitational wave events and pulsar timing array experiments,''
Phys. Rev. D \textbf{95}, no.12, 123510 (2017).
%doi:10.1103/PhysRevD.95.123510
%[arXiv:1611.06130 [astro-ph.CO]].
%103 citations counted in INSPIRE as of 29 Oct 2020

%\cite{Inomata:2017okj}
\bibitem{Inomata:2017okj}
K.~Inomata, M.~Kawasaki, K.~Mukaida, \me and T.~T.~Yanagida,
%``Inflationary Primordial Black Holes as All Dark Matter,''
Phys. Rev. D \textbf{96}, no.4, 043504 (2017).
%doi:10.1103/PhysRevD.96.043504
%[arXiv:1701.02544 [astro-ph.CO]].
%92 citations counted in INSPIRE as of 29 Oct 2020

%\cite{Inomata:2017uaw}
\bibitem{Inomata:2017uaw}
K.~Inomata, M.~Kawasaki, K.~Mukaida, \me and T.~T.~Yanagida,
%``$\mathcal O(10) M_\odot$ primordial black holes and string axion dark matter,''
Phys. Rev. D \textbf{96}, no.12, 123527 (2017).
%doi:10.1103/PhysRevD.96.123527
%[arXiv:1709.07865 [astro-ph.CO]].
%7 citations counted in INSPIRE as of 29 Oct 2020

%\cite{Kogai:2020jkq}
\bibitem{Kogai:2020jkq}
K.~Kogai and \me,
%``Escape from the swampland with a spectator field,''
Phys. Rev. D \textbf{101}, no.10, 103514 (2020)
doi:10.1103/PhysRevD.101.103514
[arXiv:2003.06753 [astro-ph.CO]].
%0 citations counted in INSPIRE as of 29 Oct 2020

%\cite{Niikura:2019kqi}
\bibitem{Niikura:2019kqi}
H.~Niikura, M.~Takada, S.~Yokoyama, T.~Sumi and S.~Masaki,
%``Constraints on Earth-mass primordial black holes from OGLE 5-year microlensing events,''
Phys. Rev. D \textbf{99}, no.8, 083503 (2019).
%doi:10.1103/PhysRevD.99.083503
%[arXiv:1901.07120 [astro-ph.CO]].
%64 citations counted in INSPIRE as of 29 Oct 2020

%\cite{Tada:2019amh}
\bibitem{Tada:2019amh}
\me and S.~Yokoyama,
%``Primordial black hole tower: Dark matter, earth-mass, and LIGO black holes,''
Phys. Rev. D \textbf{100}, no.2, 023537 (2019).
%doi:10.1103/PhysRevD.100.023537
%[arXiv:1904.10298 [astro-ph.CO]].
%19 citations counted in INSPIRE as of 29 Oct 2020

%\cite{Tada:2015noa}
\bibitem{Tada:2015noa}
\me and S.~Yokoyama,
%``Primordial black holes as biased tracers,''
Phys. Rev. D \textbf{91}, no.12, 123534 (2015).
%doi:10.1103/PhysRevD.91.123534
%[arXiv:1502.01124 [astro-ph.CO]].
%48 citations counted in INSPIRE as of 29 Oct 2020


%\cite{Mikura:2020qhc}
\bibitem{Mikura:2020qhc}
Y.~Mikura, \me and S.~Yokoyama,
%``Conformal inflation in the metric-affine geometry,''
[arXiv:2008.00628 [hep-th]].
%1 citations counted in INSPIRE as of 29 Oct 2020

%\cite{Abe:2020sqb}
\bibitem{Abe:2020sqb}
K.~T.~Abe, \me and I.~Ueda,
%``Induced gravitational waves as a cosmological probe of the sound speed during the QCD phase transition,''
[arXiv:2010.06193 [astro-ph.CO]].
%0 citations counted in INSPIRE as of 29 Oct 2020

\end{thebibliography}
	
%end 応募者の研究遂行能力及び研究環境 ====================
%begin 研究業績リスト ====================
	
	%end 研究業績リスト ====================

\input{pieces/p03_abilities_01}

%#Split: 04_rights  
%#PieceName: p04_rights
\input{pieces/p04_rights_00}
\section{4 人権の保護及び法令等の遵守への対応}
%    <<最大 1ページ>>

% s09_rights
%begin 人権の保護及び法令等の遵守への対応 ====================

\noindent
該当しない.
	
%end 人権の保護及び法令等の遵守への対応 ====================

\input{pieces/p04_rights_01}

%#Split: 99_tail
\input{pieces/hook9} % pieces
\end{document}

