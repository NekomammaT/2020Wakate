% inst_general.tex
%--------------------------------------------------------------------
% For writing instructions
%--------------------------------------------------------------------
\newcommand{\KLInstWOGeneral}[1]{%
	\noindent
 ーー ※留意事項 ーーーーーーーーーーーーーーーーーーーーーーーーーーーーーー\\
		#1\\
 ーーーーーーーーーーーーーーーーーーーーーーーーーーーーーーーーーーーーーーーー
}

\newcommand{\KLInst}[1]{%
	\noindent
	\ifthenelse{\equal{#1}{}}{%
 ーー ※留意事項 ーーーーーーーーーーーーーーーーーーーーーーーーーーーーーーー\\
	}{%
 ーー ※留意事項\textcircled{1} ーーーーーーーーーーーーーーーーーーーーーーーーーーーーーーー\\
		#1\\
		
	\noindent
 ーー ※留意事項\textcircled{2} ーーーーーーーーーーーーーーーーーーーーーーーーーーーーーーー\\
	}
}

% local variables for \GeneralInstructions ------------------------
\newcommand{\留意事項内種目名}{}			% #1
\newcommand{\留意事項内記入要領名}{}		% #2
\newcommand{\留意事項内種目名削除用}{}	% #3

\newcommand{\GeneralInstructions}[3]{%
	\ifthenelse{\equal{#1}{}}{%
		\renewcommand{\留意事項内種目名}{研究計画調書}%
	}{%
		\renewcommand{\留意事項内種目名}{#1}%
	}%
	\ifthenelse{\equal{#2}{}}{%
		\renewcommand{\留意事項内記入要領名}{作成・記入要領}%
	}{%
		\renewcommand{\留意事項内記入要領名}{#2}%
	}%
	\ifthenelse{\equal{#3}{}}{%
		\renewcommand{\留意事項内種目名削除用}{\留意事項内種目名}%
	}{%
		\renewcommand{\留意事項内種目名削除用}{#3}%
	}%
  1.作成に当たっては、\留意事項内種目名\留意事項内記入要領名 を必ず確認すること。\\
  2.本文全体は11ポイント以上の大きさの文字等を使用すること。\\
  3.各頁の上部のタイトルと指示書きは動かさないこと。\\
  4.指示書きで定められた頁数は超えないこと。なお、空白の頁が生じても削除しないこと。\\
  \textcolor{red}{5.本留意事項は、\留意事項内種目名削除用 の作成時には削除すること。(\texttt{\textbackslash JSPSInstructions}などを消す)}\\
 ーーーーーーーーーーーーーーーーーーーーーーーーーーーーーーーーーーーーーーーー
}


\newcommand{\PapersInstructions}{%
 ーー ※留意事項 ーーーーーーーーーーーーーーーーーーーーーーーーーーーーーーー\\
1. 研究業績(論文、著書、産業財産権、招待講演等)は、網羅的に記載するのではなく、\\
 本研究計画の実行可能性を説明する上で、その根拠となる文献等の主要なものを適宜記\\
 載すること。\\
2. 研究業績の記述に当たっては、当該研究業績を同定するに十分な情報を記載すること。\\
 例として、学術論文の場合は論文名、著者名、掲載誌名、巻号や頁等、発表年(西暦)、\\
 著書の場合はその書誌情報、など。\\
3. 論文は、既に掲載されているもの又は掲載が確定しているものに限って記載すること。\\
\textcolor{red}{4. 本留意事項は、\留意事項内種目名削除用 の作成時には削除すること。
 (\texttt{\textbackslash PapersInstructions}を消す)}\\
 ーーーーーーーーーーーーーーーーーーーーーーーーーーーーーーーーーーーーーーーー\\
}